%% -!TEX root = AUTthesis.tex
% در این فایل، عنوان پایان‌نامه، مشخصات خود، متن تقدیمی‌، ستایش، سپاس‌گزاری و چکیده پایان‌نامه را به فارسی، وارد کنید.
% توجه داشته باشید که جدول حاوی مشخصات پروژه/پایان‌نامه/رساله و همچنین، مشخصات داخل آن، به طور خودکار، درج می‌شود.
%%%%%%%%%%%%%%%%%%%%%%%%%%%%%%%%%%%%
% دانشکده، آموزشکده و یا پژوهشکده  خود را وارد کنید
\faculty{دانشکده ...}
% گرایش و گروه آموزشی خود را وارد کنید
\department{گرایش ...}
% عنوان پایان‌نامه را وارد کنید
\fatitle{عنوان پایان نامه-دستورالعمل و راهنمای نگارش پایان‌نامه}
% نام استاد(ان) راهنما را وارد کنید
\firstsupervisor{نام کامل استاد راهنما}
%\secondsupervisor{استاد راهنمای دوم}
% نام استاد(دان) مشاور را وارد کنید. چنانچه استاد مشاور ندارید، دستور پایین را غیرفعال کنید.
\firstadvisor{نام کامل استاد مشاور}
%\secondadvisor{استاد مشاور دوم}
% نام نویسنده را وارد کنید
\name{نام }
% نام خانوادگی نویسنده را وارد کنید
\surname{و نام خانوادگی کامل نویسنده}
%%%%%%%%%%%%%%%%%%%%%%%%%%%%%%%%%%
\thesisdate{ماه و سال}

% چکیده پایان‌نامه را وارد کنید
\fa-abstract{
در اين قسمت چكيده پایان نامه نوشته مي‌شو‌د‌.‌ چكيده بايد جامع و بيان‌كننده‌ خلاصه‌اي از اقدامات انجام‌شده باشد. در چكيده باید از ارجاع به مرجع و ذكر روابط رياضي، بيان تاريخچه و تعريف مسئله خودداري ‌شود. 
}


% کلمات کلیدی پایان‌نامه را وارد کنید
\keywords{کلیدواژه اول، ...، کلیدواژه پنجم (نوشتن سه تا پنج واژه کلیدی ضروری است)}



\AUTtitle
%%%%%%%%%%%%%%%%%%%%%%%%%%%%%%%%%%