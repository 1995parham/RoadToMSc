\chapter{راه‌حل پیشنهادی}

مساله‌ی بیان شده به صورت \lr{ILP}
مدل‌سازی می‌شود.
در \cite{Eramo2016}
مساله‌ی جایگذاری \lr{SFC}ها با هدف حداکثرسازی تعداد درخواست‌های پذیرفته شده
به صورت \lr{ILP} مدل‌سازی شده و اثبات شده است که مساله‌ی حاضر \lr{NP-Hard} می‌باشد.
مساله‌ای که در اینجا مدل‌سازی می‌شود از آن مساله پیچیده‌تر می‌باشد زیرا در نظر گرفتن \lr{VNFM}ها را نیز شامل می‌شود
بنابراین این مساله نیز \lr{NP-Hard} خواهد بود.
برای این مساله می‌توان
یک راه حل مکاشفه‌ای با زمان چند جمله‌ای
می‌توان پیشنهاد داد. این راه حل بهینه نبوده و به همین علت کارآیی آن
در سناریوهایی با مدل‌سازی بهینه مقایسه می‌شود.

یکی از راه‌حل‌های ساده مرتب کردن تمام تقاضاها براساس منابع مصرفی (پهنای باند و منابع پردازشی)
و در ادامه جایگذاری آن‌ها از تقاضای با کمترین منابع مصرفی به تقاضای با بیشترین منابع مصرفی می‌باشد.
در ادامه از تقاضا با کمترین منابع مصرفی آغاز کرده و آن را روی سرورها قرار می‌دهیم، برای این امر یک تابع ارزش‌دهی پیشنهاد می‌شود
و این جایگذاری روی سرور با بیشترین ارزش صورت می‌پذیرد.
در نهایت نگاشت لینک‌ها صورت می‌پذیرد، برای این کار نگاشت با هدف توزیع‌بار و به صورت چند مسیره صورت می‌پذیرد.