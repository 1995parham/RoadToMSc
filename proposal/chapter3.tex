\chapter{کارهای مرتبط}
در \cite{Eramo2016}
نویسندگان قصد دارند با در نظر گرفتن محدودیت ظرفیت لینک‌ها و محدودیت پردازشی نودها
بیشترین تعداد زنجیره‌ی کاکرد را بپذیرند. برای این کار یک مساله‌ی \lr{ILP}
طراحی می‌کنند و ثابت می‌کنند که این مساله \lr{NP-Hard} می‌باشد.
در این مقاله وجود \lr{VNFM} برای زنجیره‌ها در نظر گرفته نشده است.

در \cite{AbuLebdeh2017}
نویسندگان استفاده از \lr{VNFM} را مدنظر قرار داده‌اند
. در این مقاله فرض شده است که جایگذاری \lr{SFC}ها صورت گرفته است
و می‌خواهیم \lr{VNFM}ها را به گونه‌ای استقرار دهیم
که با رعایت شدن نیازمندی‌های کارآیی، هزینه‌ی عملیاتی سیستم حداقل شود.
مساله مطرح شده به صورت \lr{ILP} مدلسازی می‌شود.
این مقاله هزینه‌ی عملیاتی سیستم را تحت چهار عنوان دسته‌بندی می‌کند:
هزینه‌ی مدیریت چرخه‌ی زندگی، هزینه‌ی منابع محاسباتی، هزینه‌ی مهاجرت و هزینه‌ی بازنگاشت.
در این مقاله فرض می‌شود که هر نمونه از \lr{VNFM}ها می‌تواند به تعداد مشخصی از نمونه‌های \lr{VNF}
سرویس‌دهی کند و این سرویس‌دهی به نوع نمونه وابسته نیست.
این مقاله محدودیت‌های پردازشی و ظرفیتی را مدنظر قرار می‌دهد.

در \cite{Ghaznavi2017}
نویسندگان سه مرحله برای عملیات جایگذاری زنجیره‌های کارکرد سرویس معرفی می‌کنند:
\begin{itemize}
    \item انتخاب
    \item جابگذاری
    \item مسیریابی
\end{itemize}
در این مقاله فرض می‌شود برای هر نوع \lr{VNF}
چند مدل مختلف با مصرف منابع مختلف وجود دارند که می‌توان از آن‌ها نمونه ساخت، در این مرحله مشخص می‌شود
از کدام مدل نمونه‌سازی صورت می‌گیرد.
این مقاله جایگذاری یک \lr{SFC} را مدل‌سازی می‌کند،
در این مقاله فرض می‌شود جریان ورودی و خروجی از هر نمونه برابر بوده و در واقع
\lr{VNF} تغییری بر روی ترافیک ایجاد نمی‌کند.
در مدل‌سازی این مقاله که به صورت \lr{ILP} می‌باشد هدف کاهش هزینه در جایگذاری \lr{SFC} داده شده می‌باشد.
با در نظر گرفتن مدل‌های مختلف برای \lr{VNF}ها در این مقاله
در صورتی که نیاز به پردازش ترافیک زیادی باشد، چند نمونه از یک نوع \lr{VNF}
ساخته می‌شود و ترافیک بین آن‌ها تقسیم می‌شود.

در \cite{Yu2017}
نویسندگان برای اولین‌بار مساله‌ی \lr{Traffic Streering}
با در نظر گرفتن \lr{QoS} و \lr{Reliability}
فرمول‌بندی کرده‌اند.
این مقاله کاربرد \lr{NFV} را در شبکه‌های موبایل مدنظر قرار داده است.
در این مقاله مساله به صورت \lr{Link-Path}
مدل‌سازی شده است و فرض شده است که مسیرهای ممکن برای جایگذاری کلاس‌های ترافیکی از پیش تعیین شده‌اند.
در این مقاله منظور از کیفیت سرویس تاخیر و گذردهی کلاس‌های ترافیکی می‌باشد و 
برای فراهم آوردن قابلیت اطمینان فرض می‌شود که خرابی‌ها به صورت دلخواه بوده و در صورت خرابی‌
بخشی از پهنای باند از دست می‌رود.


در \cite{Huang2017}
نویسندگان مساله‌ی جایگذاری و مسیریابی زنجیره‌های کارکرد سرویس را به صورت توامان مدل‌سازی می‌کنند،
در این مساله نویسندگان تاثیر دو پارامتر \lr{Coordination Effect} و \lr{Traffic-Change Effect}
را نیز مدنظر قرار داده‌اند.
زمانی که چند \lr{VM} در پیاده‌سازی یک کارکرد شبکه استفاده می‌شوند
نیاز است که بین این ماشین‌های مجازی هماهنگی صورت بگیرد.
برای این هماهنگی ارتباطاتی صورت می‌گیرد که دارای سربار بوده و به این سربار
\lr{Coordination Effect} می‌گویند.
هر کارکرد شبکه می‌تواند روی ترافیک ورودی خود تاثیر گذاشته و نرخ آن را تغییر دهد
که این موضوع را با \lr{Traffic-Change Effect} بیان می‌کنند.

در \cite{Chen2017}
نویسندگان قصد دارند به صورت قطعی کیفیت سرویس را گارانتی نمایند.
این مقاله پیاده‌سازی \lr{NFV} را با استفاده از \lr{SDN} هدف قرار می‌دهد
و برای محاسبه‌ی تاخیر، تاخیر پیام‌های کنترلی \lr{SDN} و
تاخیر جابجایی بسته‌ها را در نظر می‌گیرد.
برای پیشنهاد یک راه‌حل قطعی از \lr{Network Calculus}
استفاده می‌شود که شرایط مرزی را بررسی می‌کند.
این شرایط مرزی برای پیام‌های کنترلی محاسبه شده
و از آن تاخیر مورد نظر در جابجایی بسته‌ها بدست می‌آید
که با استفاده از آن یک مساله‌ی بهینه‌سازی با هدف رعایت تاخیر بدست آمده حاصل می‌شود.

در \cite{Ma2017}
نویسندگان پیاده‌سازی \lr{NFV} با \lr{SDN}
را هدف قرار داده‌اند و جایگذاری \lr{middle box}ها
با هدف توزیع‌بار را فرمول‌بندی کرده‌اند.
در واقع \lr{middle box}ها
در این مقاله به صورت مجازی بوده و همان کارکردهای مجازی شبکه می‌باشند.
مدل‌سازی صورت گرفته به صورت \lr{node link} صورت پذیرفته است.
هدف مساله مسیریابی چند مسیره برای تقاضا به صورتی است که در آن
\lr{link load ratio} برای تمام لینک‌ها می‌نیمم شود.
این مقاله تغییر ترافیک توسط کارکردها را نیز مدنظر قرار داده است.

در \cite{Jang2017}
مساله‌ی جایگذاری زنجیره‌های کاکرد سرویس با دو هدف کاهش مصرف انرژی و افزایش نرخ جریان پذیرفته شده
مدل‌سازی می‌شود. این مدل‌سازی با توجه به معماری \lr{IETF SFC} صورت پذیرفته است.
در مدلسازی این مقاله جزئیات زیادی مورد توجه قرار گرفته است که این امر باعث پیچیده شدن
فرمول‌بندی شده است.