\chapter{تعریف مساله}

\section{مساله}
مساله
پذیرفتن بیشترین تقاضای زنجیره‌ کارکرد سرویس با در نظر گرفتن انتساب هر نمونه کارکرد مجازی شبکه به یک \lr{VNFM} می‌باشد.
در این مساله فرض می‌شود مرکز داده‌ای که مساله برای آن حل می‌شود به صورت جغرافیایی توزیع شده می‌باشد.
همانطور که در مستند \cite{ETSI-MAN} نیز آمده است، نیاز است که هر یک نمونه‌های کارکردهای مجازی شبکه
توسط حداقل یک \lr{VNFM} مدیریت شوند.
در این مساله قصد داریم مساله پذیرش تقاضاهای زنجیره‌های کارکرد سرویس را با نظر گرفتن این نیازمندی در کنار
نیازمندی‌های پردازشی و پهنای‌باند هر یک از تقاضاها حل کنیم.
در ادامه به صورت خلاصه شرایط مساله را بررسی می‌کنیم:

\begin{itemize}
    \item توپولوژی زیرساخت شامل پهنای باند لینک‌ها و ظرفیت \lr{NFVI-PoP}ها موجود است.
    \item \lr{n} تقاضای زنجیره‌ کارکرد سرویس به صورت کامل و از پیش مشخص شده داریم.
    \item هر تقاضا شامل نوع و تعداد نمونه‌های مجازی و پنهای باند لینک‌های مجازی می‌باشد.
    \item \lr{F} نوع کارکرد مجازی شبکه تعریف شده است که هر یک مقدار مشخصی از حافظه و توان پردازشی را مصرف می‌کنند.
    \item تعداد پردازنده‌هایی که به هر نمونه تخصیص می‌یابد با توجه به ترافیک ورودی نمونه مشخص می‌شود.
    \item نمونه‌ها بین زنجیره‌ها به اشتراک گذاشته نمی‌شوند.
    \item محدودیت ظرفیت لینک‌ها
    \item محدودیت توان پردازش سرورهای فیزیکی با توجه به میزان حافظه و تعداد پردازنده‌ها
    \item برای مدیریت یکدست و آسان‌تر زنجیره‌ها و در عین حال جمع‌آوری راحت‌تر خطاها، برای هر زنجیره یک \lr{VNFM} تخصیص می‌دهیم.
    \item \lr{VNFM}ها می‌توانند بین زنجیره به اشتراک گذاشته شوند.
    \item هر نمونه از \lr{VNFM}ها می‌تواند تعداد مشخصی از نمونه‌های کارکرد مجازی شبکه را سرویس دهد. 
    \item برای ارتباط میان هر نمونه از \lr{VNFM}ها و \lr{VNF}ها پهنای باند مشخصی رزرو می‌گردد.
    \item بر روی هر \lr{NFVI-PoP} حداکثر یک نمونه \lr{VNFM} مستقر می‌گردد.
    \item محدودیت تاخیر روی لینک‌های مدیریتی
\end{itemize}

اگر جایگذاری \lr{VNFM}ها به صورت غیر برنامه‌ریزی شده صورت بپذیرد
ممکن است به تاخیرهای غیرقابل تحمل منجر شده و به این ترتیب تاثیر منفی بر روی کارآیی سیستم
داشته باشد.

یکی از وظایف \lr{VNFM}ها جمع‌آوری پیام‌های خطا می‌باشد،
برای این امر نیاز است که پهنای باند کوچک اما اختصاصی به \lr{VNFM}ها
تخصیص داده شود بنابراین نمی‌توان جایگذاری آن‌ها را با روش‌های سابق و مانند سایر
کارکردهای مجازی شبکه فرض کرد.

در نظرگرفتن \lr{VNFM} همراه با \lr{VNF}ها مساله‌ی جدیدی است.
در مساله‌ای که در نظر گرفته شده است \lr{VNFM}ها می‌توانند به صورت اشتراکی
تخصیص داده شوند و این در حالی است که \lr{VNF}ها این ویژگی را ندارند.
از سوی دیگر لینک‌های مدیریتی محدودیت تاخیر دارند که این محدودیت برای لینک‌های زنجیره‌ها مدنظر قرار داده نشده است.
در واقع محدودیت تاخیر برای لینک‌های زنجیره‌ها در قیاس با تاخیری که روی \lr{VNF}ها رخ می‌دهد قابل صرف نظر کردن است و برای همین
در نظر نگرفتن آن‌ها در قیاس با تاخیر لینک‌های مدیریتی کار دور از ذهنی نمی‌باشد.
با این شرایط \lr{VNFM}ها اجزای متفاوتی در سیستم ما می‌باشند و نمی‌توان مانند سایر \lr{VNF}ها با آن‌ها برخورد کرد.
از سوی دیگر نمی‌توان \lr{VNFM}ها را در زنجیره‌ها قرار داد
زیرا زنجیره‌ها بنابر تعریفی که دارند می‌بایست یک سرویس و یک جریان داده‌ای را شبیه‌سازی کنند
و این در حالی است که وجود \lr{VNFM}ها این موضوع را نقض می‌کند.

مساله جایگذاری ماشین مجازی به استقرار ماشین‌های مجازی بر روی سرورها در مراکز داده مجازی شده می‌پردازد.
عموما ماشین‌های مجازی در یک سناریو \lr{IaaS} در اختیار کاربران قرار می‌گیرند.
در این مساله از دید فراهم کننده‌ی زیرساخت، صرفا سرورهایی که ماشین‌های مجازی بر روی آن قرار می‌گیرند مهم است که باید محدودیت منابع زیرساخت
و اهداف فراهم‌کننده را برآورده کنند.
بنابراین در مقایسه با تخصیص منابع در معماری \lr{NFV} یا \lr{SFC}،
ترتیبی که جریان سرویس باید از آن عبور کند و یا به عبارت دقیق‌تر نگاشت لینک‌های بین کارکردها
در نظر گرفته نمی‌شود.
در هر صورت اگر فرض کنیم که مسیرهای بین هردو سرور زیرساخت از قبل مشخص و دارای پهنای باند کافی است،
می‌توان از نگاشت لینک‌ها صرفنظر کرد و مساله تخصیص منابع در معماری \lr{NFV}
مشابه مساله جایگذاری ماشین مجازی خواهد شد.

\section{فرمول‌بندی}

هدف اصلی مساله پذیرش بیشترین تعداد تقاضا می‌باشد. در اینجا فرض می‌کنیم پذیرش هر تقاضا سودی منحصر به فرد خواهد داشت.
بنابراین تابع هدف به شکل زیر می‌باشد:

\begin{latin}\begin{align}
    \sum_{h=1}^{T} c_hx_h
\end{align}\end{latin}



\begin{center}\begin{latin}\begin{tabular}{|c|p{10cm}|}
    \hline
    $x_h$ & binary variable assuming the value 1 if the $h$th SFC request is accepted; otherwise its value is zero \\
    \hline
    $y_{wk}$ & the number of VNF instances of type $k$ that are used in server $w \in V_s^{PN}$ \\
    \hline
    $z^k_{vw}$ & binary variable assuming the value 1 if the VNF node $v \in \cup_{i=1}^{T} V_{i, F}^{SFC}$ is served by the VNF instance of type k in the server $w \in V_s^{PN}$ \\
    \hline
    $\bar{y}_w$ & binary varibale assuming the value 1 if VNFM on server $w \in V_s^{PN}$ is used; otherwise its value is zero\\
    \hline
    $\bar{z}_{hw}$ & binary variable assuming the value 1 if $h$th SFC is assigned to VNFM on server $w \in V_s^{PN}$\\
    \hline
\end{tabular}\end{latin}\end{center}

برای هر نود اندازه‌ی مشخصی از حافظه \lr{RAM}
در نظر گرفته می‌شود که هر نمونه‌ی کارکرد با توجه به نوع آن مقدار مشخصی از این حافظه را مصرف می‌کند.
\begin{latin}\begin{align}
    \sum_{k=1}^F y_{wk} memory(k) + \bar{y_w} \bar{memory} \le N_{ram}^{PN}(w)
    \quad
    \forall w \in V_s^{PN}
\end{align}\end{latin}

برای هر نود تعداد مشخصی از پردازنده در نظر گرفته می‌شود که هر نمونه‌ی کارکرد با توجه به نوع آن مقدار مشخصی از این حافظه را مصرف می‌کند.
\begin{latin}\begin{align}
    \sum_{k=1}^F y_{wk} core(k) + \bar{y_w} \bar{core} \le N_{core}^{PN}(w)
    \quad
    \forall w \in V_s^{PN}
\end{align}\end{latin}

اگر \lr{VNF}, \lr{v}
توسط \lr{VNF instance} نوع \lr{k}
روی سرور \lr{w} سرویس شود می‌بایست
\lr{VNF instance} نوع \lr{k}
روی سرور \lr{w} فعال شود.
اشتراک گذاری \lr{VNF}ها پشتیبانی نمی‌گردد.
\begin{latin}\begin{align}
\sum_{v \in \cup_{i=1}^T V_{i, F}^{SFC}} z_{vw}^k \le y_{wk}
\quad
\forall w \in V_s^{PN}, \forall k \in [1,\ldots, F]
\end{align}\end{latin}

اگر تقاضای \lr{h}ام پذیرفته شده باشد
می‌بایست تمام \lr{VNF node}های آن‌
سرویس شده باشند.
یک \lr{VNF} حداکثر یکبار سرویس داده شود.
\begin{latin}\begin{align}
    x_h = \sum_{k=1}^{F} \sum_{w \in V_{s}^{PN}} z_{vw}^{k}
    \quad
    \forall v \in V_{h,F}^{SFC}, \forall h \in [1,\ldots, T]
\end{align}\end{latin}

اگر تقاضای \lr{h}ام پذیرفته شده باشد
می‌بایست توسط یک \lr{VNFM} سرویس شده باشد.
\begin{latin}\begin{align}
    x_h = \sum_{w \in V_{s}^{PN}} \bar{z}_{hw}
    \quad
    \forall h \in [1,\ldots, T]
\end{align}\end{latin}

اگر \lr{SFC}، \lr{i}
توسط \lr{VNFM} روی سرور \lr{w}
سرویس شود می‌بایست \lr{VNFM} سرور \lr{w}
فعال شود.
\begin{latin}\begin{align}
    \bar{z}_{hw} \le \bar{y}_w
    \quad
    \forall w \in V_{s}^{PN}, \forall h \in [1,\ldots, T]
\end{align}\end{latin}

محدودیت ظرفیت سرویس‌دهی \lr{VNFM}
\begin{latin}\begin{align}
    \sum_{i=1}^{T} z_{iw} \le capacity
    \quad
    \forall w \in V_{s}^{PN}
\end{align}\end{latin}

\begin{center}\begin{latin}\begin{tabular}{|c|p{10cm}|}
    \hline
    $\tau^{(u,v)}_{ij}$ & binary variable assuming the value 1 if the virual link $(u,v)$ is routed on the physical network link $(i,j)$\\
    \hline
    $\bar{\tau}^{v}_{ij}$ & binary variable assuming the value 1 if the managemnt of VNF node $v$ is routed on the physical network link $(i,j)$\\
    \hline
\end{tabular}\end{latin}\end{center}

\lr{Flow Conservation}
\begin{latin}\begin{align}
    \sum_{(i,j) \in E^{PN}} \tau_{ij}^{(u,v)} - \sum_{(j,i) \in E^{PN}} \tau_{ji}^{(u,v)} = \sum_{k=1}^{F} z_{ui}^{k} - \sum_{k=1}^{F} z_{vi}^{k} \nonumber \\
    \forall i \in V_{S}^{PN}, (u,v) \in E_{h}^{SFC}, h \in [1,\ldots, T]
\end{align}\end{latin}
\begin{latin}\begin{align}
    \sum_{(i,j) \in E^{PN}} \bar{\tau}_{ij}^{v} - \sum_{(j,i) \in E^{PN}} \bar{\tau}_{ji}^{v} = \sum_{k=1}^{F} z_{vi}^{k} - \bar{z}_{hi} \nonumber \\
    \forall i \in V_{S}^{PN}, v \in V_{h, F}^{SFC}, h \in [1,\ldots, T]
\end{align}\end{latin}

محدودیت ظرفیت لینک‌ها
\begin{latin}\begin{align}
    \sum_{v \in \cup_{i=1}^{T} V_{i,F}^{SFC}} \bar{\tau}_{ij}^{v} * \bar{bandwidth} + \sum_{(u,v) \in \cup_{i=1}^{T} E_{i}^{SFC}} \tau_{ij}^{(u,v)} * bandwidth(u,v) \le C_{ij}
\end{align}\end{latin}