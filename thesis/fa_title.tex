%% -!TEX root = AUTthesis.tex
% در این فایل، عنوان پایان‌نامه، مشخصات خود، متن تقدیمی‌، ستایش، سپاس‌گزاری و چکیده پایان‌نامه را به فارسی، وارد کنید.
% توجه داشته باشید که جدول حاوی مشخصات پروژه/پایان‌نامه/رساله و همچنین، مشخصات داخل آن، به طور خودکار، درج می‌شود.
%%%%%%%%%%%%%%%%%%%%%%%%%%%%%%%%%%%%
% دانشکده، آموزشکده و یا پژوهشکده  خود را وارد کنید
\faculty{دانشکده مهندسی کامپیوتر و فناوری اطلاعات}
% گرایش و گروه آموزشی خود را وارد کنید
\department{شبکه‌های کامپیوتری}
% عنوان پایان‌نامه را وارد کنید
\fatitle{زنجیره‌سازی کارکردهای مجازی سرویس شبکه با لحاظ محدودیت منابع مدیریتی}
% نام استاد(ان) راهنما را وارد کنید
\firstsupervisor{دکتر بهادر بخشی}
%\secondsupervisor{استاد راهنمای دوم}
% نام استاد(دان) مشاور را وارد کنید. چنانچه استاد مشاور ندارید، دستور پایین را غیرفعال کنید.
% \firstadvisor{نام کامل استاد مشاور}
%\secondadvisor{استاد مشاور دوم}
% نام نویسنده را وارد کنید
\name{پرهام}
% نام خانوادگی نویسنده را وارد کنید
\surname{الوانی}
%%%%%%%%%%%%%%%%%%%%%%%%%%%%%%%%%%
\thesisdate{شهریور ۱۳۹۸}

% چکیده پایان‌نامه را وارد کنید
\fa-abstract{
    مساله‌ی مجازی سازی توابع شبکه سعی دارد توابع شبکه را به صورت مجازی در شبکه جایگذاری نمایند و در ادامه
    با برقراری ارتباط میان آن‌ها سرویس‌هایی را فراهم آورد.
    یکی از مسائل در این روش پذیرش سرویس‌ها و قرار دادن آن‌ها بر روی زیرساخت است
    که در کارهایی زیادی به آن پرداخته شده است ولی یکی از اجزا معماری مجازی سازی کارکردهای
    شبکه بخش مدیریتی است که می‌بایست در کنار سرویس‌ها بر روی زیرساخت مستقر شود.
    در این رساله ما قصد داریم جایگذاری سرویس‌ها با لحاظ منابع مدیریتی را
    مدل‌سازی و حل نماییم.
}


% کلمات کلیدی پایان‌نامه را وارد کنید
\keywords{مجازی سازی کارکردهای شبکه، زنجیره‌سازی کارکردهای مجازی سرویس شبکه،بهینه‌سازی، بهینه‌سازی خطی صحیح}



\AUTtitle
%%%%%%%%%%%%%%%%%%%%%%%%%%%%%%%%%%