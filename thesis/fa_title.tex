%% -!TEX root = AUTthesis.tex
% در این فایل، عنوان پایان‌نامه، مشخصات خود، متن تقدیمی‌، ستایش، سپاس‌گزاری و چکیده پایان‌نامه را به فارسی، وارد کنید.
% توجه داشته باشید که جدول حاوی مشخصات پروژه/پایان‌نامه/رساله و همچنین، مشخصات داخل آن، به طور خودکار، درج می‌شود.
%%%%%%%%%%%%%%%%%%%%%%%%%%%%%%%%%%%%
% دانشکده، آموزشکده و یا پژوهشکده  خود را وارد کنید
\faculty{دانشکده مهندسی کامپیوتر و فناوری اطلاعات}
% گرایش و گروه آموزشی خود را وارد کنید
\department{گرایش شبکه‌های کامپیوتری}
% عنوان پایان‌نامه را وارد کنید
\fatitle{زنجیره‌سازی کارکردهای مجازی سرویس شبکه با در نظر گرفتن محدودیت منابع مدیریتی}
% نام استاد(ان) راهنما را وارد کنید
\firstsupervisor{دکتر بهادر بخشی}
%\secondsupervisor{استاد راهنمای دوم}
% نام استاد(دان) مشاور را وارد کنید. چنانچه استاد مشاور ندارید، دستور پایین را غیرفعال کنید.
% \firstadvisor{نام کامل استاد مشاور}
%\secondadvisor{استاد مشاور دوم}
% نام نویسنده را وارد کنید
\name{پرهام}
% نام خانوادگی نویسنده را وارد کنید
\surname{الوانی}
%%%%%%%%%%%%%%%%%%%%%%%%%%%%%%%%%%
\thesisdate{شهریور ۱۳۹۸}

% چکیده پایان‌نامه را وارد کنید
\fa-abstract{
    در روش سنتی استقرار سرویس شبکه، ترافیک کاربر باید از تعدادی کارکرد شبکه به ترتیب معینی عبور کند تا یک مسیر پردازش ترافیک ایجاد شود.
    در حال حاضر این کارکردها به صورت سخت‌افزاری به یکدیگر متصل هستند و ترافیک با استفاده از جداول مسیریابی به سمت آن‌ها هدایت می‌شود.
    چالش اصلی این روش در این است که استقرار و تغییر ترتیب کارکردها دشوار است.
    دو فناوری برای پاسخ گویی به این چالش ها مطرح شد:
    مجازی‌‍سازی کارکرد شبکه (\lr{NFV}) و زنجیره‌سازی کارکرد سرویس (\lr{SFC}).
    با استفاده از مجازی‌سازی کارکردهای شبکه و اجرای آن‌ها بر روی سرورهای استاندارد با توان بالا، امکان اجرای کارکردها بر روی سخت افزارهای عمومی را فراهم کرده است
    تا نیاز به تجهیزات سخت افزاری خاص منظوره کاهش یابد.
    از طرف دیگر \lr{SFC} امکان تعریف زنجیره کارکردها را ارائه می‌کند
    که ایجاد و انتخاب مسیرهای متفاوت برای پردازش ترافیک به صورت پویا و بدون ایجاد تغییر در زیرساخت فیزیکی را امکان‌پذیر می‌کند.
    با توجه به این فناوری‌ها، مسائل تحقیقاتی جدیدی مطرح شدند که از مهم‌ترین آن‌ها می توان تخصیص منابع بهینه به سرویس درخواستی کاربر را نام برد.
    یکی از چالش‌های مهم در زنجیره‌سازی کارکرد سرویس چگونگی جایگذاری کارکرد‌ها در شبکه زیرساخت می‌باشد که تا به حال پژوهش‌های زیادی در این حوزه انجام شده است.
    یکی دیگر مسائلی که در معماری \lr{NFV} مطرح است چگونگی مدیریت و مانیتورینگ کارکردهای مجازی می‌باشد. تا به حال این دو مساله در کنار یکدیگر مورد مطالعه قرار نگرفته‌اند
    و این در حالی است که برای ارائه سرویس‌هایی با کیفیت مناسب نیاز است که مدیریت و مانیتورینگ بر روی آن‌ها صورت بگیرد.
    در این رساله ما به بررسی همین مساله می‌پردازیم.
    در اولین گام مساله فوق به صورت خطی صحیح فرمول‌بندی شده و در چهارچوب \lr{CPLEX} پیاده‌سازی می‌شود.
    از آنجایی که این مساله \lr{NP-Hard} می‌باشد نیاز است برای حل آن در زمان مناسب از یک الگوریتم مکاشفه‌ای با پیچیدگی چند جمله‌ای استفاده شود.
    این رساله الگوریتمی با زمان چند جمله‌ای برای این مساله پییشنهاد می‌دهد و در نهایت آن را با مساله‌ی بهینه مقایسه می‌کند.  
    در نتیجه‌ی این مقایسه الگوریتم پیشنهادی در زمان اجرای کمتر جوابی نزدیک الگوریتم بهینه
    ارائه می‌کند. الگوریتم پیشنهادی در کنار جایگذاری زنجیره‌ها، نگاشت منابع مدیریتی را نیز انجام می‌دهد.
}


% کلمات کلیدی پایان‌نامه را وارد کنید
\keywords{مجازی سازی کارکردهای شبکه، زنجیره‌سازی کارکردهای مجازی سرویس شبکه،بهینه‌سازی، بهینه‌سازی خطی صحیح}



\AUTtitle
%%%%%%%%%%%%%%%%%%%%%%%%%%%%%%%%%%
