\chapter{نتیجه‌گیری و کارهای آینده}

در این فصل نتیجه‌گیری و جمع‌بندی پژوهش‌ها و فعالیت‌های انجام شده
در این رساله آورده شده است.
همچنین با توجه به کارهای انجام شده،
کارهای آینده برای پیشبرد این پژوهش برشمرده شده‌اند.

\section{جمع‌بندی و نتیجه‌گیری}

همانطور که پیشتر بیان شد پژوهش‌های زیادی به مسائل جایگذاری
زنجیره‌های کارکرد سرویس پرداخته‌اند
اما هیچ یک از آن‌ها نیازمندی مدیریتی برای این زنجیره‌ها همانطور که در
\cite{ETSI-MAN}
بیان شده است را در نظر نگرفته‌اند.

از آنجایی که مانیتورینگ و مدیریت زنجیره‌های در مراکز داده اهمیت زیادی دارد،
نیاز است تا در هنگام جایگذاری زنجیره‌ها منابع مدیریتی آن‌ها را نیز مدنظر قرار داد.
در این پژوهش تلاش شد مساله‌ای جامع و نزدیک به واقعیت برای جایگذاری توامان زنجیره‌ها و منابع مدیریتی
آن‌ها طرح و روشی برای حل آن ارائه شود.

در این پژوهش، در ابتدا یک مدل بهینه‌سازی
\lr{MILP}
برای مساله‌ی طرح شده ارائه شد که هدف آن بیشینه کردن میزان سود حاصل از جایگذاری زنجیره‌ها
و تخصیص منابع مدیریتی به آن‌ها بود.
در این مدل سعی شد تا تمامی سیاست‌های یک مرکز داده‌ای در جایگذاری کارکردهای مجازی
و منابع مدیریتی آن‌ها مدنظر باشد.
با توجه به اینکه مساله‌ی طرح شده یک مساله‌ی \lr{NP-Hard}
است و پیدا کردن راه‌حل بهینه از نظر محاسباتی
و زمانی مقدور نیست.
در این پژوهش یک الگوریتم اکتشافی ارائه شد که دارای دو بخش اصلی
نگاشت زنجیره و نگاشت منابع مدیریتی است.

در بخش نگاشت زنچیره‌ها ابتدا زنجیره‌ها بر اساس سودشان مرتب می‌شوند
در ادامه بر اساس الگوریتم
\cite{Bari2015}
جایگذاری شده و در نهایت بعد از نگاشت زنجیره
\lr{VNFM}
ان انتخاب می‌گردد. در این انتخاب سعی می‌شود از \lr{VNFM}
با ظرفیت خالی و منابع آزاد استفاده شود.

در نهایت در قسمت ارزیابی این الگوریتم با حالت بهینه و الگوریتم
\cite{Bari2015}
مقایسه می‌شود که نتیجه نشان می‌دهد سود حاصل از جایگذاری با استفاده از این الگوریتم
به طور میانگین ۹۰ درصد الگوریتم بهینه بوده و ۵ درصد نسبت به \cite{Bari2015}
افزایش دارد.
در نظر داشته باشید که این الگوریتم منابع مدیریتی را نیز مدنظر قرار می‌دهد
که پیشتر در پژوهش‌ها اشاره‌ای به آن‌ها نشده است.
این الگوریتم با بیشتر شدن تعداد زنجیره‌ها کارآایی بیشتری از خود نشان می‌دهد
چرا که قسمت مرتب‌سازی با افزایش تعداد زنجیره‌ها باعث می‌گردد زنجیره‌هایی با سود بیشتر جایگذاری شوند و به سود کلی عملیات کمک می‌کند.

\subsection{کارهای آینده}

در این پژوهش تلاش شد تا ابعاد مختلف جاسازی زنجیره‌ها در نظر گرفته شود.
با این وجود مواردی وجود دارد که در نظر گرفتن آن‌ها می‌تواند باعث دقیق‌تر شدن مساله
و نزدیک شدن آن به واقعیت شود.
در ادامه این موارد مرور می‌شوند:

\begin{itemize}
    \item در نظر گرفتن نیاز به \lr{VNFO} برای مدیریت تعاملات زنجیره‌ها
    \item در نظر گرفتن نیاز ارتباط \lr{NVFI-PoP} با \lr{VIM}
    \item به اشتراک گذاری کارکردهای مجازی و در نظر گرفتن هزینه‌ی گواهی برای آن‌ها
\end{itemize}