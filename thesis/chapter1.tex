\chapter{مقدمه}

در سال‌های اخیر دو تکنولوژی شبکه‌های نرم‌افزارمحور و مجازی‌سازی شبکه بسیار مورد توجه قرار گرفته‌اند.
پیشتر در ارائه سرویس‌های شبکه، از سخت‌افزارهای اختصاصی که توسط سازندگان اختصاصی ارائه می‌شد و به آن‌ها
\lr{middle box}
گفته می‌شد استفاده می‌گشت.
تنوع و تعداد رو به افزایش سرویس‌های جدیدی که توسط کاربران تقاضا می‌گردد
باعث هزینه‌های زیاد برای خرید و نگهداری
\lr{middle box}‌ها
توسط اپراتورها شده است.
به تازگی فراهم آورندگان شبکه
شروع به حرکت به سوی مجازی‌سازی و نرم‌افزاری کردن بسترهای شبکه کرده‌اند،
به این ترتیب آن‌ها قادر خواهند بود
سرویس‌های نوآورانه‌ای به کاربران ارائه بدهند.
این روند به سرویس دهندگان اجازه می‌دهد که ارائه سرویس‌های دلخواه‌شان وابسته به سخت‌افزارهای اختصاصی نباشد و 
هزینه‌های راه‌اندازی و نگهداری فراهم آوردندگان سرویس را کاهش می‌دهد.
با نرم‌افزاری سازی کارکردها، وابستگی آن‌ها به سخت افزار اختصاصی کاهش یافته و به سرعت می‌توان آن‌ها را افزایش/کاهش مقیاس داد.
مجازی‌سازی کارکردهای شبکه و زنجیره‌سازی کارکرد سرویس‌ راهکاری‌هایی هستند که برای همین منظور پیشنهاد شده‌اند.

ایده‌ی اصلی مجازی‌سازی توابع شبکه جداسازی تجهیزات فیزیکی شبکه از کارکردهایی می‌باشد که
بر روی آن‌ها اجرا می‌شوند.
به این معنی که یک کارکرد شبکه مانند دیوار آتش می‌تواند بر روی سرورهای
\lr{HVS}\footnote{\lr{High Volume Server}}
به عنوان یک نرم‌افزار ساده مستقر شود.
با این روش یک سرویس می‌تواند با استفاده از کارکردهای مجازی شبکه‌ای که می‌توانند به صورت نرم‌افزاری پیاده‌سازی شده
و روی یک یا تعدادی سرور استاندارد فیزیکی اجرا شوند، استقرار یابد.
کارکردهای مجازی شبکه‌ای می‌توانند در مکان‌های مختلف بازمکان‌یابی یا نمونه‌سازی شوند بدون آنکه
نیاز به خریداری و نصب تجهیز جدیدی باشد.
\cite{Mijumbi2016}

\section{تعریف صورت مساله}
مساله‌ی جاسازی کارکردهای مجازی شبکه یکی از چالش‌های مهم در تخصیص منابع به زنجیره‌های کارکرد سرویس می‌باشد.
مساله جاسازی کارکردهای مجازی شبکه به دو زیر مساله‌ی نگاشت گره‌های مجازی و نگاشت یال‌های تقسیم می‌شود که می‌بایست
به صورت توامان در نظر گرفته شوند.

البته محدودیت‌های زیادی وجود دارد که باید هنگام نگاشت در نظر گرفته شود. منابع فیزیکی انتخاب شده
از شبکه زیرساخت باید نیازمندی‌های کارکرد شبکه مجازی را تامین کنند به عنوان مثال
قدرت پردازشی کارکرد‌های مجازی باید کمتر یا مساوی با قدرت پردازشی گره فیزیکی باشد که نگاشت
روی آن انجام شده است.

علاوه بر این، مجموعه‌ای از محدودیت‌ها وجود دارد که مختص زنجیره‌های کارکرد سرویس می‌باشد.
یکی از این موارد وجود
\lr{VNFM}
در این شبکه‌های می‌باشد که به علت اهمیت میزان تاخیر ارتباط بین کاکرد مجازی شبکه و
\lr{VNFM}
می‌بایست در مکان مناسبی جایابی شود بنابراین زیرمساله‌ی جدیدی به مساله‌ی اصلی اضافه می‌شود.

\section{اهمیت مساله}
مساله‌ی جاسازی زنجیره‌های کارکرد سرویس از اهمیت زیادی برخوردار است و پروژهش‌های زیادی
بر روی آن صورت پذیرفته است.
در کنار این جاسازی مساله مدیریت و مانیتورینگ این زنیجره‌ها نیز مطرح است
که این پروژه برای اولین بار این موضوع را نیز مدنظر قرار داده است که باعث می‌شود اهمیت
مساله دو چندان شود.

\section{نوآوری}
ایده‌ی اصلی این پروژهش، ارائه‌ی یک راه‌حل جامع و کامل که تمامی ابعاد مساله‌ی جاسازی
زنجیره‌های کارکرد سرویس را در بربگیرد، است.
در واقع در این مساله علاوه بر در نظر گرفتن ابعاد اصلی مساله‌ی جاسازی مکانیزم کنترل پذیرش،
قابل اعمال بودن راه‌حل به توپولوژی‌های مختلف و وجود محدودیت‌های گره و یال ابعاد دیگری نیز در نظر گرفته شده است.
به علت وجود \lr{VNFM}
به عنوان یک گره خاص و اهمیت تاخیر اتصالات کارکرد مجازی شبکه و \lr{VNFM}
یک مرحله جایابی و نگاشت به مساله‌ی اصلی اضافه شده است.
در ادامه محدودیت‌هایی برای اتصالات بین کارکرهای مجازی شبکه و \lr{VNFM}ها
در نظر گرفته شده است و فرض شده است برای مدیریت تعداد مشخصی از کارکرهای مجازی نیاز به تهیه
مجوز با هزینه‌ای مشص است.

\section{ساختار گزارش}
در ادامه معماری \lr{NFV}
را معرفی می‌کنیم
و به چالش‌هایی که در \lr{MANO} وجود دارد می‌پردازیم.
در فصل سوم کارهای مرتبط مرور می‌شوند و در فصل چهارم مساله تعریف شده بیان می‌گردد. در فصل پنجم
در مورد راه‌حل پیشنهادی برای مساله بحث خواهد شد.
در آخر در فصل ششم راه‌حل پیشنهادی ارزیابی می‌گردد.